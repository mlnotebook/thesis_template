\subsection{Evaluation Strategies}
\label{sec:3-seg_eval}

This section can hold the methods used for evaluating the experiments performed in the this Chapter.

\subsubsection{Some Metric}

Equations can be used in a number of ways. The \gls{dsc} is used as an example below.

\begin{enumerate}[label=\textbf{\arabic*}]
\item Simple Display Equation: 
% This '%' is necessary to reduce the gap between the text and equation - but it can be useful to keep it if equations spill over pages (thus the skip has not been hardcoded)
\begin{equation}
    \mathrm{DSC} = \frac{2 \left|\textbf{A} \cap \textbf{B}\right|}{\left|\textbf{A}\right| + \left|\textbf{B}\right|} \label{eq:simple_display_equation}   
\end{equation}

\item Annotated Display Equation: `falign` adds a title to the left (or right) of the equation. Uses the table cell notation `\&\&` to separate the left, middle and right parts.
% 
\begin{flalign}
    \text{Dice Similarity Coefficient:\quad} &&\mathrm{DSC} = \frac{2 \left|\textbf{A} \cap \textbf{B}\right|}{\left|\textbf{A}\right| + \left|\textbf{B}\right|} \label{eq:dsc}&&    
\end{flalign}

\item Multi-Equation: `align` can align multiple equations by using the `\&` symbol on the equals sign and `\textbackslash\textbackslash` where the newline begins. Multiple labels can be used - one for each equation.
% 
\begin{align}
    y &= (x+1)^2 \label{eq:aligned_1}\\
    y &= (x^2 + 2x + 1) \label{eq:aligned_2}
\end{align}

\end{enumerate}
