\usepackage[english]{babel}
\usepackage{csquotes,xpatch}
\usepackage[backend=biber,style=numeric-comp,maxnames=3,minnames=3,giveninits=true,sorting=nyt,url=false]{biblatex}
\DeclareNameAlias{author}{family-given}
\DeclareNameAlias{default}{family-given}
\addbibresource{meta/0-references.bib}

\usepackage[acronym]{glossaries}
\usepackage{fancyref}

% Create a new reference format for Listings (of source code) - allows the use of \fref{lst:listing_label} and creates a 'List of Listings' in the table of contents.

\newcommand*{\fancyreflstlabelprefix}{lst}

% Main format
\frefformat{main}{\fancyreflstlabelprefix}
    {\MakeUppercase{\freflstname}\fancyrefdefaultspacing#1#2}
\Frefformat{main}{\fancyreflstlabelprefix}
    {\MakeUppercase{\Freflstname}\fancyrefdefaultspacing#1#2}
\Frefformat{main}{\fancyreflstlabelprefix}{Listing~#1}
\frefformat{main}{\fancyreflstlabelprefix}{listing~#1}

% Vario format
\frefformat{vario}{\fancyreflstlabelprefix}
    {\MakeUppercase{\freflstname}\fancyrefdefaultspacing#1#2}
\Frefformat{vario}{\fancyreflstlabelprefix}
    {\MakeUppercase{\Freflstname}\fancyrefdefaultspacing#1#2}
\Frefformat{vario}{\fancyreflstlabelprefix}{Listing~#1 on page #2}
\frefformat{vario}{\fancyreflstlabelprefix}{listing~#1 on page #2}
\makeglossaries
